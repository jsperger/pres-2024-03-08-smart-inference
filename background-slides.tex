\documentclass[aspectratio=169, professionalfonts]{beamer}

\usepackage{graphicx}
\usepackage{subfig}
\usepackage{longtable}
\usepackage{wrapfig}
\usepackage{rotating}
\usepackage[normalem]{ulem}
\usepackage{amsmath}
\usepackage{amssymb}
\usepackage{capt-of}

\usepackage{hyperref} \hypersetup{ colorlinks=true,
	linkcolor=blue, filecolor=magenta, urlcolor=cyan, pdftitle={Overleaf Example},
	pdfpagemode=FullScreen}

%% Personal macros across multiple projects
\input{components/combined-macros.tex}
%% Paper-specific macros

\usepackage[style=verbose-ibid,backend=biber, isbn=false, url=false, doi=false,
	eprint=false, dashed=false]{biblatex}
\addbibresource{my-biblatex-library.bib}

% Assumed directory structure is structure both VK and its associated files
% live in ./components/vk-theme
\usepackage{components/vk-theme/beamerthemeVK}

\author{John Sperger}
\date{March 8\textsuperscript{th} 2024}
\title{Value Function Inference}
\subtitle{Background and Notation Reference}
\begin{document}
\maketitle
\begin{frame}[label={sec:org00d2d44}]{Notation}

	Let $[K]$ denote the set $\{1, \ldots, K \}$ for a positive integer $K$. Assume
	the data is comprised of iid replicates:

	\begin{equation}
		\left\{{\covarrv}_{\obsindex},\, {\armrv}_{\obsindex},\,
		{\resprv}_{\obsindex} \right\}_{\obsindex = 1}^{\maxobsindex}
	\end{equation}

	where
	\vfill \pause

	\begin{itemize}
		\item $\covarrv_{\obsindex} \in \covarspace \subseteq
			      \symbb{R}^{\dimarmparam}$ denotes the covariates (contexts)

		\item $\armrv_{ \obsindex} \in \armspace$ denotes the treatment arm (arm,
		      intervention, action)

		\item $\resprv_{ \obsindex} \in \symbb{R}$ denotes the response (reward)
	\end{itemize}

	\vfill \pause

	We'll use the potential outcome framework and denote the potential outcome 	$\po(\armobs)$

\end{frame}


\begin{frame}{Operator notation}
	\begin{itemize}
		\item $\covarrv_1, \ldots, \covarrv_{\maxobsindex}$ is an iid random sample from a
		      fixed but unknown distribution $\Pop$
		\item $\genericfun$ is a generic parametric function indexed by $\theta \in \Theta$
		\item $\hat{\theta} \in \Theta$ is a random variable constructed from the sample
		      $\covarrv_1, \ldots, \covarrv_{\maxasymindex}$
	\end{itemize}
	\vfill \pause

	$\Pop$ denotes the probability measure:
	\begin{equation*}
		\Pop \genericfun(\covarrv; \hat{\theta}) = \int \genericfun(\covarobs;\,\widehat{\theta}) \dv
		\Pop(\covarobs))
	\end{equation*}

	\vfill \pause $\Pn$ denotes the corresponding empirical measure:

	\begin{equation*}
		\Pn \genericfun(\covarrv; \hat{\theta}) = \asymindex^{-1}\sum_{i =
			1}^{\asymindex} \genericfun(\covarobs_{i};\,\hat{\theta})
	\end{equation*}
	\vfill \pause

	$\leadsto$ denotes convergence in distribution.

	\vfill
\end{frame}

\subsection{Treatment Policy Notation}

\begin{frame}{Treatment Policy Notation}
	A policy (aka DTR, treatment rule) $\pol$ maps contexts to actions $\pol: \covarspace \mapsto \armspace$.

	\vfill \pause
	The expected response given
	the covariates and arm assignment $\E[\resprv \vert \covarobs, \armobs =
			\armindex]$is given by $\meanfun(\covarobs, \armobs = \armindex) \equiv \meanfunk(x)$
	for some function $\meanfun \in \funclass$.

	\vfill \pause
	Define the value of a policy $\pol$ as
	\begin{equation}
		\val(\pol) = \E_{\covarrv}[\po(\armobs = \pol(\covarrv))]
	\end{equation}

	\vfill \pause
	An optimal policy $\optpol$ is any policy that satisfies
	\begin{equation}
		\val(\optpol) \geq \val(\pol) \qquad \text{for all }\pol \in \polset
	\end{equation}
\end{frame}

\begin{frame}{Value Functions}
	\begin{enumerate}
		\item Conditional Value of the estimated optimal policy

		      $$\val(\polhatn) = \E_{\covarrv}[\polhatn(\covarrv) \vert \polhatn] = \Pop \polhatn(\covarrv)$$

		      \vfill \pause

		\item (Expected) Value of an estimated optimal policy
		      $$\E_{\covarrv}[\polhatn(\covarrv)]$$

		      \vfill \pause

		\item Value of the optimal policy
		      $$\val(\optpol) = \E_{\covarrv}[\optpol(\covarrv)] = \Pop \optpol(\covarrv)$$
	\end{enumerate}

	\vfill \pause

	Will estimators of these functions asymptotically equivalent?

	\vfill \pause
	Brainstorm a scenario for each of the value functions where that estimand
	would 	make the most sense.
\end{frame}


\begin{frame}{The Estimated Optimal Policy}
	Denote the estimated optimal policy $\polhatn$, and to simplify life assume
	that there is a unique maximizer for every covariate value, or more formally

	$$\text{For all} \covarrv \in \covarspace \text{ there exists an arm }k^*
		\text{ such that } \meanfun_{\armindex^*}(\covarrv) > \max_{\armindex \neq
			\armindex^*} \meanfunk(\covarrv)$$

	\vfill \pause
	This assumption ensures that the optimal policy is deterministic:
	$$\polopt(\covarobs) = \argmax_{\armindex \in [\maxarmindex]}\meanfunk(\covarobs)$$

	The estimated optimal policy
\end{frame}


\begin{frame}{Operator Notation Exercise}
	Rewrite the following in operator notation:
	\vfill
	\begin{itemize}

		\item Conditional Value of the estimated optimal policy

		      $$\E_{\covarrv}[\polhatn(\covarrv) \vert \polhatn] = ?$$

		\item Value of the optimal policy
		      $$\E_{\covarrv}[\optpol(\covarrv)] = ?$$

		\item Estimated Value of the optimal policy

		      $$\valhatn(\polhatn(\covarrv)) = ?$$
	\end{itemize}
	\vfill

\end{frame}

\begin{frame}{Operator Notation Exercise}
	Rewrite the following in operator notation:
	\vfill
	\begin{itemize}

		\item Conditional Value of the estimated optimal policy
		      \pause
		      $$\E_{\covarrv}[\polhatn(\covarrv) \vert \polhatn] = \Pop \polhatn(\covarrv)$$
		      \pause
		\item Value of the optimal policy
		      \pause
		      $$\E_{\covarrv}[\optpol(\covarrv)] = \Pop \optpol(\covarrv)$$
		      \pause
		\item Estimated Value of the estimated optimal policy
		      \pause
		      $$\valhatn(\polhatn(\covarrv)) = \Pn \val(\polhatn(\covarrv))$$
	\end{itemize}
	\vfill

\end{frame}


\begin{frame}{Operator Notation Exercise}
	Rewrite the following in operator notation:
	\vfill
	\begin{itemize}

		\item Conditional Value of the estimated optimal policy

		      $$\E_{\covarrv}[\polhatn(\covarrv) \vert \polhatn] = \Pop \polhatn(\covarrv)$$

		\item Value of the optimal policy
		      $$\E_{\covarrv}[\optpol(\covarrv)] = \Pop \optpol(\covarrv)$$
	\end{itemize}
	\vfill

\end{frame}


\begin{frame}{The Estimated Optimal Policy}
	Denote the estimated optimal policy $\polhatn$, and to simplify life assume
	that there is a unique maximizer for every covariate value, or more formally

	$$\text{For all} \covarrv \in \covarspace \text{ there exists an arm }k^*
		\text{ such that } \meanfun_{\armindex^*}(\covarrv) > \max_{\armindex \neq
			\armindex^*} \meanfunk(\covarrv)$$

	\vfill \pause
	This assumption ensures that the optimal policy is deterministic:
	$$\polopt(\covarobs) = \argmax_{\armindex \in [\maxarmindex]}\meanfunk(\covarobs)$$

	The estimated optimal policy
\end{frame}

\begin{frame}{Treatment Policy Notation}
	A policy (aka DTR, treatment rule) $\pol$ maps contexts to actions $\pol: \covarspace \mapsto \armspace$.

	\vfill \pause
	The expected response given
	the covariates and arm assignment $\E[\resprv \vert \covarobs, \armobs =
			\armindex]$is given by $\meanfun(\covarobs, \armobs = \armindex) \equiv \meanfunk(x)$
	for some function $\meanfun \in \funclass$.

	\vfill \pause
	Define the value of a policy $\pol$ as
	\begin{equation}
		\polval = \val(\pol) \doteq \E_{\covarrv}[\po(\armobs = \pol(\covarrv))]
	\end{equation}

	\vfill \pause
	An optimal policy $\optpol$ is any policy that satisfies
	\begin{equation}
		\val(\optpol) \geq \val(\pol) \qquad \text{for all }\pol \in \polset
	\end{equation}

	\vfill \pause
	Denote the estimated optimal policy $\polhatn$, and for conciseness, assume
	that there is a unique maximizer for every covariate value.

\end{frame}



\appendix
\printbibliography

\end{document}
%%% Local Variables:
%%% mode: LaTeX
%%% TeX-master: t
%%% End:
